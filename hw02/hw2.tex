\documentclass[10pt]{article}
\usepackage[utf8]{inputenc}

\usepackage[margin=1in]{geometry}
\usepackage{amsmath}
\usepackage{amssymb}
\usepackage{amsthm}
\usepackage{mathtools}

\usepackage{hyperref}
\hypersetup{
  colorlinks   = true, %Colours links instead of ugly boxes
  urlcolor     = black, %Colour for external hyperlinks
  linkcolor    = blue, %Colour of internal links
  citecolor    = blue  %Colour of citations
}

\usepackage{color}
\usepackage{colortbl}
\definecolor{deepblue}{rgb}{0,0,0.5}
\definecolor{deepred}{rgb}{0.6,0,0}
\definecolor{deepgreen}{rgb}{0,0.5,0}
\definecolor{gray}{rgb}{0.7,0.7,0.7}

%%%%%%%%%%%%%%%%%%%%%%%%%%%%%%%%%%%%%%%%%%%%%%%%%%%%%%%%%%%%%%%%%%%%%%%%%%%%%%%%

\theoremstyle{definition}
\newtheorem{problem}{Problem}
\newcommand{\E}{\mathbb E}
\newcommand{\R}{\mathbb R}
\DeclareMathOperator{\Var}{Var}
\DeclareMathOperator*{\argmin}{arg\,min}
\DeclareMathOperator*{\argmax}{arg\,max}

\newcommand{\trans}[1]{{#1}^{T}}
\newcommand{\loss}{\ell}
\newcommand{\w}{\mathbf w}
\newcommand{\x}{\mathbf x}
\newcommand{\y}{\mathbf y}
\newcommand{\ltwo}[1]{\lVert {#1} \rVert}


\usepackage{listings}

% Default fixed font does not support bold face
\DeclareFixedFont{\ttb}{T1}{txtt}{bx}{n}{12} % for bold
\DeclareFixedFont{\ttm}{T1}{txtt}{m}{n}{12}  % for normal

% Python style for highlighting
\newcommand\pythonstyle{\lstset{
language=Python,
basicstyle=\ttm,
otherkeywords={self},             % Add keywords here
keywordstyle=\ttb\color{deepblue},
emph={MyClass,__init__},          % Custom highlighting
emphstyle=\ttb\color{deepred},    % Custom highlighting style
stringstyle=\color{deepgreen},
frame=tb,                         % Any extra options here
showstringspaces=false            % 
stepnumber=1,
numbers=left
}}

\lstnewenvironment{python}[1][]
{
    \pythonstyle
    \lstset{#1}
}
{}

%%%%%%%%%%%%%%%%%%%%%%%%%%%%%%%%%%%%%%%%%%%%%%%%%%%%%%%%%%%%%%%%%%%%%%%%%%%%%%%%

\begin{document}

\begin{center}
    {
\Large
CSCI046 Homework 2: Runtime Analysis
}

    %\vspace{0.1in}
%CSCI046, Mike Izbicki

    \vspace{0.1in}
    \textbf{DUE: Thursday, 6 February beginning of class}

    \vspace{0.1in}
\end{center}

\vspace{0.25in}
\noindent
Name: 

\noindent
\rule{\textwidth}{0.1pt}
\vspace{0.15in}

\noindent
\textbf{Collaboration policy:} 
You are allowed to use any resources you would like to complete this assignment,
and you are encouraged to work in teams.
Remember, learning the material is your responsibility,
so collaborate in a way that will help you learn.
\vspace{0.15in}

\noindent
\textbf{Grading note:} 
Each of these problems contains several subproblems,
and there are typically more subproblems than the point value of the problem.
You will lose 1 point for each incorrect subproblem.
If this would result in a negative score, then you get zero for the problem.
\vspace{0.15in}


%\newpage
\begin{problem}
    (3 points)
    Simplify the following expressions:

\begin{enumerate}
    \item $O\bigg((n^2 + n\log n)(n^3 + \log n)\bigg)$
    \vspace{1.5in}
\item $\Omega\bigg((3.45n + n)(\log n^2)\bigg)$
    \vspace{1.5in}
\item $\Theta\bigg(n(1 + \log n) + n^{3.2} + \log 2^n\bigg)$
\end{enumerate}
\end{problem}

\newpage
\begin{problem}
(3 points)
    %For each row below, circle $O$ if $f=O(g)$ and $\Omega$ if $f=\Omega(g)$.
    %If $f=\Theta(g)$, then circle both $O$ and $\Omega$.
    Complete each equation below by adding the symbol $O$ if $f=O(g)$, $\Omega$ if $f=\Omega(g)$, or $\Theta$ if $f=\Theta(g)$.  
    The first row is completed for you as an example.

{\renewcommand{\arraystretch}{4.4}
\begin{tabular}{c c c c c c}
    & f(n) &~\hspace{0.5in}~$ $~\hspace{0.5in}~& g(n) &\\
    \hline
    & $1$ & ~\hspace{0.5in}~$=$~\hspace{0.5in}~  & $O(n)$ &  &\\
    \arrayrulecolor{gray}\hline
    & $3 n\log n$ & ~\hspace{0.5in}~$=$~\hspace{0.5in}~  & $n^2$ &  &\\
    \arrayrulecolor{gray}\hline
    & $1$ & ~\hspace{0.5in}~$=$~\hspace{0.5in}~  & $1/n$ &  &\\
    \arrayrulecolor{gray}\hline
    & $\log_2 n$ & ~\hspace{0.5in}~$=$~\hspace{0.5in}~  & $\log_3 n$ &  &\\
    \arrayrulecolor{gray}\hline
    & $n^{42}$ & ~\hspace{0.5in}~$=$~\hspace{0.5in}~  & $42^n$ &  &\\
    \arrayrulecolor{gray}\hline
    & $5\cdot10^{30}$ & ~\hspace{0.5in}~$=$~\hspace{0.5in}~  & $\log n$ &  &\\
    \arrayrulecolor{gray}\hline
    & $\log n$ & ~\hspace{0.5in}~$=$~\hspace{0.5in}~  & $\log (n^2)$ &  &\\
    \arrayrulecolor{gray}\hline
    & $2^n$ & ~\hspace{0.5in}~$=$~\hspace{0.5in}~  & $3^n$ &  &\\
    \arrayrulecolor{gray}\hline
    & $n!$ & ~\hspace{0.5in}~$=$~\hspace{0.5in}~  & $n^2$ &  &\\
    \arrayrulecolor{gray}\hline
    & $\log n$ & ~\hspace{0.5in}~$=$~\hspace{0.5in}~  & $(\log n)^2$ &  &\\
    \arrayrulecolor{gray}\hline

    %& $O(1)$ & or & $O(n)$ & or & equal\\
    %& $O(n\log n)$ & or & $O(n^2)$ & or & equal\\
    %& $\Theta(1)$ & or & $\Theta(1/n)$ & or & equal\\
    %& $\Omega(\log_2 n)$ & or & $\Omega(\log_3 n)$ & or & equal\\
    %& $O(n^{42})$ & or & $O(42^n)$ & or & equal\\
    %& $\Theta(5\cdot10^{30})$ & or & $\Theta(\log n)$ & or & equal\\
    %& $\Omega(\log n)$ & or & $\Omega(\log (n^2))$ & or & equal\\
    %& $O(2^n)$ & or & $O(3^n)$ & or & equal\\
    %& $\Theta(n!)$ & or & $\Theta(n^2)$ & or & equal\\
    %& $\Omega(\log n)$ & or & $\Omega((\log n)^2)$ & or & equal\\
\end{tabular}
}

\end{problem}


\newpage
\begin{problem}
(2 points)
Answer the questions below based on the following python code:
\begin{python}
for i in range(100):
    print('a')
    for j in range(50):
        print('b')
        for k in range(25):
            print('c')
    for j in range(25):
        print('b')
    for k in range(25):
        print('d')
for j in range(100):
    print('d')
\end{python}
\begin{enumerate}
    \item What is the exact number of times that the letter \texttt{a} will be printed?
        \vspace{1.5in}
    \item What is the exact number of times that the letter \texttt{b} will be printed?
        \vspace{1.5in}
    \item What is the exact number of times that the letter \texttt{c} will be printed?
        \vspace{1.5in}
    \item What is the exact number of times that the letter \texttt{d} will be printed?
        \vspace{1.5in}
\end{enumerate}
\end{problem}

\newpage
\begin{problem}
(2 points)
Answer the questions below based on the following python code:
\begin{python}
for i in range(n**2):
    print('a')
    for j in range(n/2):
        print('b')
        for k in range(int(math.sqrt(n))):
            print('c')
        print('b')
    print('b')
for i in range(n):
    print('b')
for i in range(100):
    print('d')
    print('a')
\end{python}
\begin{enumerate}
    \item What is the asymptotic number of times that the letter \texttt{a} will be printed? 
        (Use $\Theta$ notation.)
        \vspace{1.5in}
    \item What is the asymptotic number of times that the letter \texttt{b} will be printed?
        (Use $\Theta$ notation.)
        \vspace{1.5in}
    \item What is the asymptotic number of times that the letter \texttt{c} will be printed?
        (Use $\Theta$ notation.)
        \vspace{1.5in}
    \item What is the asymptotic number of times that the letter \texttt{d} will be printed?
        (Use $\Theta$ notation.)
        \vspace{1.5in}
\end{enumerate}
\end{problem}

\newpage
\begin{problem}
    (0.5 points extra credit)
    Prove the following: $\log(n!) = \theta(n\log n)$,
\end{problem}
\end{document}
